\section{FURTHER RESEARCH}
\label{sec:Further}

Once we completed the design, building and research of the prototype actuated manipulator, we evaluated the potential issues that arose from what we had done. Below are listed a few options for improvements and/or directions to travel with the robot manipulator:
\begin{itemize}
\item Free Rotation
\begin{itemize}
\item Ideally, the whole apparatus needs to be free rotating similarly to the Velcro apparatuses so that the R-one can grab and object and turn to face any direction and then move.
\item Our prototype does not do this because of a concern of how to deal with the wires of having two counter-rotating servos. One way to solve this is to use a metal contact strip around the entirety of the robot that is in constant contact with the servo terminals in order to allow for free rotation.
\item Another option would be to run counter-rotation disks with the use of gears and mount the servo to the top or bottom.
\item This counter-rotation, however, forces problems of how to attach the manipulator to the robot. Having two disks rotate that cannot be connected to anything leads to a very flimsy or minimalist approach to design. The best solution to this we see is to continue using the bump skirt clips that have proven themselves from both the Velcro manipulator as well as our prototype actuated manipulator. 
\end{itemize}
\item The ring slot would work better if it was a slot, instead of an oval, this needs only a pin on an arm from the servo to drive it, instead of a large bulky screw.
\item The prototype requires a spacer between the moving ring and the bottom, stationary ring. This is evident in the video in the repository. Something thin with a cutout (like the ring the servo is mounted to) would probably work best.
\item One servo, clearly, works, however, using two may be better and provide a stronger grip. We didn't have time to experiment with this. 
\begin{itemize}
\item However, in order to prevent stripping/burning out of the servos, servo savers can be used. We were unable to find them small enough to fit for the application we require
\end{itemize}
\item Larger claws, or more spaced out claws would allow for gripping of larger objects and is a possibility for future adaptation. 
\item Making the claws thicker would help the robots grip one another, opening another area of research. 
\item An issue that we see is the design of the objects that are gripped by the robot, as they must have vertical, cylindrical shafts in order to be accurately grabbed by the robot. No known fix for this has been explored. It may just be that this is how things "in the robot world" must be constructed. 
\item Other ideas
\begin{itemize}
\item the moving ring could be an inverted gear (teeth on the inside) and the drive shaft comes through all layers and spins this. (it is also a possibility for part of the counter-rotating claw idea.)
\item explore different ideas for actuation using drive shafts, direct drives, and gears.
\end{itemize}
\item Incorporating a force sensor would be highly beneficial for the R-one manipulator project, because it would allow the R-one to detect when it has grabbed an object based from if the actuator has exceeded a certain force threshold, as long as the servo is not at its maximum wavelength. 
\end{itemize}