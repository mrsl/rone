\section{DESIGN}
\label{sec:Design}

After initially sitting down at the drawing board, we went through many iterations of design options with this prototype actuated manipulator. 
One of the many decisions we had to make was the best way to grab items. We decided on the servo-actuated gripper because of its relatively simple design and expandability.
There are many improvements and other areas of research that can be expanded on with this design and we will explore a few of those in Section \ref{sec:Further}. 
Next we decided to use three claw levels with only one actuated claw disk because it provides the best stability of the item that it is grabbing. The three claws provide three points of contact with the item, such as a pen, and thus hold the item straight up and down. 
If we were to use a two claw design, we foresaw that this could potentially cause the item to become pinched between the claws and twist. Thus, the three claw design was chosen. 

Next, we decided to use 18 claws for the robots because the servo has a limited range of motion, usually only 180 to 190 degrees, and since we knew we would be mounting the servo to the collar itself, we knew that we technically only had close to 30 degrees of motion in the disk. 

ABS plastic was chosen because of both its strength and lightweight properties. It is stronger than wood, and this is beneficial because the servo exerts a good amount of torque, which would cause the small pieces of wood to break. An added bonus of the ABS that we chose was that it has a very low coefficient of friction on the textured side, which is great for the movement of the pieces.

\begin{comment}
\begin{figure}[h]
\begin{center}
\includegraphics[scale=0.55]{./figs/}
\end{center}
\caption{}
\label{fig:}
\end{figure}
\end{comment}